%==============================================
\section{Fusion power and momentum conservation}
\label{appendix:fusion_power}

Deuterium-tritium fusion reactions result from inelastic collisions, for which momentum is conserved, not energy. The total kinetic energy release for a single reaction amounts to $\hat E_0 = 17.59$MeV. So as to evaluate the fraction of energy carried out by the neutron, relativistic corrections have to be taken into account. The method is detailed below. \\

Let's admit that it is sufficient to account for relativistic corrections for neutrons only\footnote{Within the classical framework, one would predict that the neutron carries 4 fifth of the total energy, i.e. about $\hat E_n \approx 14.07$MeV. At this energy, it turns out that the velocity of neutrons reaches approximately 17\% of the speed of light ($v_n/c = (2.10^6e\hat E_n/m_n)^{1/2}/c \approx 0.17$). Relativistic corrections cannot be ignored in this case. Conversely, heavier $\alpha$ particles move at about 4\% the speed of light. Relativistic corrections can be neglected for them.}. $\alpha$ particles will be treated within the classical framework.
Momentum conservation then reads:
\begin{equation}
    m_n \gamma_n v_n = m_\alpha v_\alpha
    \label{eq:conserv_momentum}
\end{equation}
with $\gamma_n = (1-v_n^2/c^2)^{-1/2}$ the Lorentz factor for the neutrons. In the limit $v_n/c \ll1$,  $\gamma_n$ can be Taylor expanded, so that eq.\ref{eq:conserv_momentum} can be recast as follows:
\begin{equation}
    u \left( 1+\frac{\epsilon}{2}\; u^2 \right) 
    = 1
    \label{eq:conserv_momentum2}
\end{equation}
with $u \doteq v_n/(\mu v_\alpha)$, $\mu \doteq m_\alpha/m_n$ and $\epsilon \doteq \mu (v_\alpha/c)\ll1$. It is sufficient to look for perturbative solutions of the form: 
$$ u = u_0 + u_1 \;\;\; \textrm{with} \;\; u_1\ll u_0 $$
The leading order yields $u_0=1$. At next order, one readily finds:
$$ u_1 = -\frac{\epsilon^2}{2} \; u_0^3 $$
So that, in the end:
\begin{equation*}
    v_n \simeq \mu v_\alpha \left( 1-\frac{\epsilon^2}{2}\right)
\end{equation*}
where we recall that $\mu=m_\alpha/m_n$.
The kinetic energy of the neutron can then be expressed as a function of the one of the $\alpha$ particle:
\begin{eqnarray*}
    E_n &=& m_nc^2(\gamma_n-1) \approx 
    \frac{1}{2} m_nv_n^2\left[ 1+ \frac{3}{4}\frac{v_n^2}{c^2}\right] \nonumber \\
    &\approx& \mu\; E_\alpha \left( 1-\frac{\epsilon^2}{4}\right)
\end{eqnarray*}

The kinetic energy of the $\alpha$ particle $E_\alpha$ can then be expressed as a function of the total energy $E_0$ ($E_0 =E_\alpha + E_n$). Elementary algebra leads to the following relation:
\begin{equation}
    E_\alpha^2 - \frac{2(1+\mu)}{\mu^2}E_{n0}\, E_\alpha + \frac{2}{\mu^2}\; E_{n0}E_0 = 0
\end{equation}
where $E_{n0} = m_nc^2$ stands for the mass energy of the neutron. We have used the relation $\epsilon^2 = 2\mu\; E_\alpha/E_{n0}$. The only acceptable solution is\footnote{Notice that, at leading order in $E_0/E_{n0}\ll1$, this solution simply reduces to $E_\alpha \approx E_0/(1+\mu)$.}:
\begin{equation}
    E_\alpha = \frac{1+\mu}{\mu^2}
    \left( 1 - \sqrt{1-\frac{2\mu^2}{(1+\mu)^2}\frac{E_0}{E_{n0}}}\right)\; E_{n0}
\end{equation}
When performing the numerical calculation, it should be warned that $\mu\neq 4$. 
Indeed, the mass of the $\alpha$ particle is slightly less than the sum of its components (actually, this mass difference $\Delta m \approx 0.0187\; m_p$ is the one which leads to the energy release of the D-T fusion reaction $E_0 = \Delta m\;c^2$). The masses can be found in reference \cite{Wesson2004}. In particular, $m_n \approx (1+0.001378)\, m_p$ and:
%$$
%    m_n \approx (1+0.001378)m_p \;\;\; ; \;\;\; 
%    m_\alpha = 2(m_n+m_p) \approx (1-0.027404)m_p \approx 3.967\; m_n
%$$
$$
    \mu \doteq \frac{m_\alpha}{m_n} \approx (1-0.027404)\; \frac{m_p}{m_n} \approx 3.967
$$
With these data, one finally obtains $E_0/E_\alpha \approx 4.94$ and $\hat E_\alpha \approx 3.56\,$MeV, which corresponds to the published value.  